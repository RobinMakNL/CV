% Colors
\definecolor{pagebackground}{HTML}{FFFFFF}
\definecolor{sidebarbackground}{HTML}{f3f3f4}
\definecolor{headercolor}{HTML}{6c6c6d}
\definecolor{sidebartitle}{HTML}{24272B}
\definecolor{sidebartext}{HTML}{6c6c6d}
\definecolor{bodytitle}{HTML}{24272B}
\definecolor{bodytext}{HTML}{6c6c6d}
\definecolor{bulletcolor}{HTML}{6c6c6d}
\definecolor{linkcolor}{HTML}{3E78B2}
\definecolor{linecolor}{HTML}{24272B}
\definecolor{name}{HTML}{FFFFFF}
\definecolor{jobtitle}{HTML}{FFFFFF}
\RequirePackage[pagecolor=pagebackground, nopagecolor=pagebackground]{pagecolor}

% Heading styles
\titleformat{\section}
{\color{bodytitle}\fontsize{16pt}{16pt}\sffamily\addfontfeature{LetterSpace=2}}
{}{0pt}{}[{\color{linecolor}\titlerule[1pt]}]

\titleformat{\subsection}
{\color{sidebartitle}\fontsize{16pt}{16pt}\sffamily\addfontfeature{LetterSpace=2}}
{}{0pt}{}[{\color{linecolor}\titlerule[1pt]}]

\titleformat{\subsubsection}
{\color{bodytitle}\fontsize{12pt}{12pt}\sffamily\addfontfeature{LetterSpace=2}}
{}{0pt}{}

\titlespacing*{\section}{0pt}{0.5cm}{0.4cm}
\titlespacing*{\subsection}{0pt}{0.5cm}{0.4cm}
\titlespacing*{\subsubsection}{0pt}{0.5cm}{0.1cm}

% Defining font styles
\newcommand{\sname}{\color{name}\fontsize{36pt}{36pt}\sffamily\addfontfeature{LetterSpace=2}\MakeUppercase}
\newcommand{\sjobtitle}{\color{jobtitle}\fontsize{22pt}{22pt}\sffamily\addfontfeature{LetterSpace=5}\MakeUppercase}

% Defining page styles
\pagestyle{fancy}
\fancypagestyle{firstpagestyle}{
    \fancyhf{}
    \renewcommand{\headrulewidth}{0pt}
    \renewcommand{\footrulewidth}{0pt}
    \fancyhead[C]{%
        \begin{tikzpicture}[remember picture,overlay]
            \node [rectangle, fill=headercolor, anchor=north west, minimum width=\paperwidth+1cm, minimum height=7.5cm] (box) at ($(current page.north west)+(-1cm,1cm)$) {};
            \node [rectangle, fill=sidebarbackground, anchor=north west, minimum width=7cm, minimum height=\paperheight-2cm] (box) at ($(current page.north west)+(1cm,-1cm)$) {};
        \end{tikzpicture}
    }
}

\fancyhf{}
\renewcommand{\headrulewidth}{0pt}
\renewcommand{\footrulewidth}{0pt}
\fancyhead[C]{%
    \begin{tikzpicture}[remember picture,overlay]
        \node [rectangle, fill=sidebarbackground, anchor=north west, minimum width=7cm, minimum height=\paperheight-2cm] (box) at ($(current page.north west)+(1cm,-1cm)$) {};
    \end{tikzpicture}
}

% Layout for name, job title and sidebar
\setcolumnwidth{6cm,11.5cm}
\setlength{\columnsep}{1cm}
\columncolor{sidebartext}[0]
\columncolor{bodytext}[1]

\newcommand{\makeprofile}{
    \thispagestyle{firstpagestyle}
    \begin{paracol}{2}
        \switchcolumn[1]
        \begin{spacing}{2}
            \vspace*{0.5cm}
            {\sname\name}

            \medskip
            {\sjobtitle\jobtitle}
        \end{spacing}

        \ifthenelse{\equal{\profilepic}{}}{}{
            \begin{tikzpicture}[remember picture,overlay]
                \node[circle,minimum width=4cm, path picture={
                \node at (path picture bounding box.center){\includegraphics[width=4cm]{\profilepic}};}] at ($(current page.north west)+(4.5cm,-4cm)$) {};
            \end{tikzpicture}
        }
    \end{paracol}
    \vspace*{1cm}
}

% Margins
\RequirePackage[left=1.5cm, right=1cm, top=1.5cm, bottom=1.5cm, footskip=0.5cm, headheight=0.5cm]{geometry}

% Personal skills bar
\newcommand\perskills[1]{
    \ifthenelse{\equal{#1}{}}{\booltrue{perempty}}{\renewcommand{\perskills}{%
        \foreach [count=\i] \x/\y in {#1}{
            {\x}\\
            \ifthenelse{\equal{\thecolumn}{0}}{
                \progressbar[width=.9\columnwidth,roundnessa=1pt, ticksheight=0, heightr=1, linecolor=linecolor, filledcolor=sidebartext, borderwidth=1pt, emptycolor=sidebarbackground]{\y}\par\smallskip
                }{
                \progressbar[width=.9\columnwidth,roundnessa=1pt, ticksheight=0, heightr=1, linecolor=bodytext, filledcolor=bodytext, borderwidth=1pt, emptycolor=pagebackground]{\y}\par\smallskip
            }
        }
    }}
}

% Professional skills bar
\newcommand\proskills[1]{
    \ifthenelse{\equal{#1}{}}{\booltrue{proempty}}{\renewcommand{\proskills}{%
        \foreach [count=\i] \x/\y in {#1}{
            {\x}\\
            \ifthenelse{\equal{\thecolumn}{0}}{
                \progressbar[width=.9\columnwidth,roundnessa=1pt, ticksheight=0, heightr=1, linecolor=linecolor, filledcolor=sidebartext, borderwidth=1pt, emptycolor=sidebarbackground]{\y}\par\smallskip
                }{
                \progressbar[width=.9\columnwidth,roundnessa=1pt, ticksheight=0, heightr=1, linecolor=bodytext, filledcolor=bodytext, borderwidth=1pt, emptycolor=pagebackground]{\y}\par\smallskip
            }
        }
    }}
}

% Command for drawing language skill circles
% Adapted from AltaCV Template: https://www.overleaf.com/latex/templates/altacv-template/trgqjpwnmtgv
\newcommand\langskills[1]{%
    \ifthenelse{\equal{#1}{}}{\booltrue{langempty}}{\renewcommand{\langskills}{%
        \renewcommand{\arraystretch}{1.4}
        \foreach [count=\i] \x/\y in {#1}{%
            \begin{tabular*}{\columnwidth}{@{}L{.4\columnwidth}L{.6\columnwidth}}
                {\x} &
                \foreach \z in {1,...,5}{%
                    {\ifnumgreater{\z}{\y}{\color{bodytext!30}}{\color{bodytext}}\small{\faCircle}}}\\
            \end{tabular*}
        }
    }}
}
